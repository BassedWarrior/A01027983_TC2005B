\documentclass{article}

\title{Understanding GDD}
\author{Fausto Jiménez de la Cuesta Vallejo}
\date{\today}

\begin{document}
\maketitle

The main point of a Game Design Document (GDD) is to serve as a main point of 
reference for all people involved in the development of a game. It is meant to 
guide newcommers as they integrate themselves, and also to anchor existing 
developers to the main vision of the game. This prevents the game concept from 
getting too big to handle, while also providing flexibility for it to grow and
evolve.

It should be updated often as to reflect the progress and the changes made 
while developing the game. And it should be fun to read for the developers, 
as it is meant to inspire them to work on the game, not bore them. It is 
important that it is referenced and used often to maintain team cohesion and 
focus on the tasks that pertain to the game rather than spending time on ideas 
which fall outside the scope of the game.

The main points of a GDD are:

\begin{description}
    \item[Game Concept] Explain the nature of the game. What it's about, and 
        what the core gameplay is.
    \item[Game Genre] This dictates the type of game mechanics that will be 
        used in the game, as 2 games from different genres but same game 
        concept can feel very different.
    \item[Game Mechanics] These are the game mechanics that a user will 
        interact with, and most times are what effectively makes a game stand 
        out from the rest. Includes both the User Experience (UX) and 
        User Interface (UI).
    \item[Milestones] Helps give the development team a realistic forecast of 
        how the development cycle is expected to go, how far off they are from 
        reaching their goals, and finally, from finishing the game.
\end{description}

In our GDD template, we are considering the folliwing sections:

\begin{itemize}
    \item Game Design
    \item Technical
    \item Level Design
    \item Development
    \item Graphics
    \item Sounds/Music
    \item Schedule
\end{itemize}

Of which I will be focusing on the Development and Sounds/Music sections.

\end{document}
